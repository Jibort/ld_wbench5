\documentclass{article}
\usepackage[utf8]{inputenc}
\usepackage[catalan]{babel}
\usepackage{hyperref}
\usepackage{array}
\usepackage{longtable}
\usepackage{booktabs}
\usepackage{ulem}
\usepackage{etoolbox}
\usepackage{listings}
\usepackage[table]{xcolor}
\definecolor{lightgray}{rgb}{0.9, 0.9, 0.9}
\usepackage{float}
\lstset{
  basicstyle=\small\ttfamily,
  breaklines=true,
  columns=flexible
}

\linespread{1.0}
\setlength{\parskip}{0.5ex} 

\setlength{\LTleft}{-30pt}  % Permet que la taula comenci 20pt a l'esquerra del marge
\setlength{\LTright}{-30pt}  % Permet que la taula s'estengui 20pt a la dreta del marge

% Reduir el tamany de la font o ajustar les columnes
\AtBeginEnvironment{longtable}{\scriptsize}

\newcolumntype{L}[1]{>{\raggedright\arraybackslash}p{#1}}
\newcolumntype{C}[1]{>{\centering\arraybackslash}p{#1}}

% Establim la mida de font per a totes les taules
\AtBeginEnvironment{longtable}{\footnotesize}

\title{Descripció tècnica de classes}
\author{Jordi Ibort Quintana}
\date{2025/04/15 dt.}

\begin{document}

\maketitle

\section{Descripció tècnica de classes}

\subsection{Interfície `LdDisposableIntf'}
\begin{itemize}
    \item \underline{Localització}: lib/03\_core/interfaces/ld\_disposable\_intf.dart
    \item \underline{Descripció}: Interfície per a les classes que requereixen de la implementació del mètode `dispose()'.
\end{itemize}

\rowcolors{2}{lightgray}{white}
\begin{longtable}{@{}L{1.25cm}L{4.5cm}L{9.5cm}@{}}
\toprule
\textbf{Tipus} & \textbf{Element} & \textbf{Descripció} \\
\midrule
Abstracte & void \textbf{dispose}() & Allibera els recursos fets servir per la classe.\\
\bottomrule
\end{longtable}

\subsection{Interface `LdTagIntf'}
\begin{itemize}
    \item \underline{Localització}: lib/03\_core/intf/ld\_tag\_intf.dart
    \item \underline{Descripció}: Interfície funcional pels objectes amb identificador únic 'tag'.
\end{itemize}

\rowcolors{2}{lightgray}{white}
\begin{longtable}{@{}L{1.25cm}L{4.5cm}L{8.75cm}@{}}
\toprule
\textbf{Tipus} & \textbf{Element} & \textbf{Descripció} \\
\midrule
Abstracte & String get \textbf{tag} & Retorna el tag únic de l'objecte. \\
Abstracte & set \textbf{tag}(String pTag)  & Estableix el tag únic de l'objecte.  \\
Abstracte  & void String \textbf{baseTag}() & Retorna la base del tag a fer servir en cas que no es proporcioni cap. \\
\bottomrule
\end{longtable}

\subsection{Mixin `LdTagMixin'}
\begin{itemize}
    \item \underline{Localització}: lib/03\_core/mixin/ld\_tag\_mixin.dart
    \item \underline{Implementa}: LdDisposableIntf
    \item \underline{Descripció}: Mixin per a la gestió dels tag únic de cada instància de classe.
\end{itemize}

\rowcolors{2}{lightgray}{white}
\begin{longtable}{@{}L{1.25cm}L{4.5cm}L{8.75cm}@{}}
\toprule
\textbf{Tipus} & \textbf{Element} & \textbf{Descripció} \\
\midrule
Getter & String get \textbf{tag} & Retorna el \textit{tag} únic de l'objecte. \\
Setter & set \textbf{tag} (String \textit{pTag}) & Estableix el tag únic de l'objecte. \\
\bottomrule
\end{longtable}

\subsection{Classe `LdTagBuilder'}
\begin{itemize}
    \item \underline{Localització}: lib/03\_core/ld\_tag\_builder.dart
    \item \underline{Descripció}: Responsable de formar tags únics correctes per a les instàncies de les classes.
\end{itemize}

\rowcolors{2}{lightgray}{white}
\begin{longtable}[H]
\begin{tabular}{@{}L{1.25cm}L{4.5cm}L{8.75cm}@{}}
\toprule
\textbf{Tipus} & \textbf{Element} & \textbf{Descripció} \\
\midrule
Estàtic & String \textbf{newViewTag}(String pTag) & Retorna el següent tag únic per a vistes. \\
Estàtic & String \textbf{newWidgetTag}(String pTag) & Retorna el següent tag únic per a widgets. \\
Estàtic & String \textbf{newModelTag}(String pTag) & Retorna el següent tag únic per a models. \\
Estàtic & String \textbf{newCtrlTag}(String pTag) & Retorna el següent tag únic per a controladors. \\
\bottomrule
\end{tabular}
\end{longtable}

\subsection{Classe `LdMap\texorpdfstring{$<$T$>$}{<T>}'}
\begin{itemize}
    \item \underline{Localització}: lib/10\_tools/ld\_map.dart
    \item \underline{Descripció}: Generalització d'un mapa amb clau String i valor del tipus especificat `T'.
\end{itemize}

\rowcolors{2}{lightgray}{white}
\begin{longtable}[H]{@{}L{1.25cm}L{4.5cm}L{8.75cm}@{}}
\toprule
\textbf{Tipus} & \textbf{Element} & \textbf{Descripció} \\
\midrule
@map & @ T? operator \textbf{[]}(Object? key) & Equivalent a l'operador `[]' de la classe \textit{map}. \\
@map & @ operator \textbf{[]=}(String key, T value) & Equivalent a l'operador `[]=' de la classe \textit{map}. \\
@map & @ \textbf{addAll}(Map$<$String, T$>$ other) & Equivalent al mètode \textit{addAll()} de la classe \textit{map}. \\
Funció & LdMap$<$T$>$ \textbf{addAllAndBack} (Map$<$String, T$>$ other) & Equivalent a addAll() però retornant la instància de \textit{LdMap}. \\
@map & @ \textbf{addEntries} (Iterable$<$MapEntry$<$String, T$>>$ newEntries) & Equivalent al mètode \textit{addEntries()} de la classe \textit{map}. \\
@map & @ Map$<$RK, RV$>$ \textbf{cast}$<$RK, RV$>$ & Equivalent al mètode \textit{cast()} de la classe \textit{map}. \\
@map & @ void \textbf{clear}() & Equivalent al mètode \textit{clear()} de la classe \textit{map}. \\
... & \textbf{... resta de la classe \textit{map}} & \\
\bottomrule
\end{longtable}

\subsection{Classe `LdBindings$<$T$>$'}
\begin{itemize}
    \item \underline{Localització}: lib/03\_core/ld\_bindings.dart
    \item \underline{Descripció}: Repositori d'instàncies 'LdTagMixin' com a dependències disponibles.
\end{itemize}

\rowcolors{2}{lightgray}{white}
\begin{table}[H]
\footnotesizes
\begin{tabular}{@{}L{1.25cm}L{4.5cm}L{8.75cm}@{}}
\toprule
\textbf{Tipus} & \textbf{Element} & \textbf{Descripció} \\
\midrule
Estàtic & String \textbf{newViewTag}(String pTag) & Retorna el següent tag únic per a vistes. \\
Estàtic & String \textbf{newWidgetTag}(String pTag) & Retorna el següent tag únic per a widgets. \\
Estàtic & String \textbf{newModelTag}(String pTag) & Retorna el següent tag únic per a models. \\
Estàtic & String \textbf{newCtrlTag}(String pTag) & Retorna el següent tag únic per a controladors. \\
\bottomrule
\end{tabular}
\end{table}

\end{document}